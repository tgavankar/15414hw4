% You should title the file with a .tex extension (hw1.tex, for example)
\documentclass[11pt]{article}

\usepackage{amsmath}
\usepackage{amssymb}
\usepackage{fancyhdr}
\usepackage{graphicx}

\oddsidemargin0cm
\topmargin-2cm         %I recommend adding these three lines to increase the
\textwidth16.5cm   %amount of usable space on the page (and save trees)
\textheight23.5cm  

\newcommand{\question}[2] {\vspace{.25in} \hrule\vspace{0.5em}
\noindent{\bf #1: #2} \vspace{0.5em}
\hrule \vspace{.10in}}
\renewcommand{\part}[1] {\vspace{.10in} {\bf (#1)}}

\newcommand{\myname}{Tanay Gavankar}
\newcommand{\myandrew}{tgavanka@andrew.cmu.edu}
\newcommand{\myhwnum}{4}

\setlength{\parindent}{0pt}
\setlength{\parskip}{5pt plus 1pt}

\pagestyle{fancyplain}
\lhead{\fancyplain{}{\textbf{HW\myhwnum}}}          % Note the different brackets!
\rhead{\fancyplain{}{\myname\\ \myandrew}}
\chead{\fancyplain{}{15-414}}

\begin{document}

\medskip                            % Skip a "medium" amount of space
                                   % (latex determines what medium is)
                                   % Also try: \bigskip, \littleskip




\thispagestyle{plain}
\begin{center}                      % Center the following lines
{\Large 15-414 Assignment \myhwnum} \\
\myname \\
\myandrew \\
Due: 11/14/12 \\
\end{center}




\question{1}{Classic Propositional Logic}
\part{a}

    \begin{tabular}{|c|c|c|c|c|}
        \hline
        $x$ & $y$ & $z$ & $(x \vee y) \rightarrow z$ & $(x \rightarrow z) \wedge (y \rightarrow z)$ \\ \hline
        0 & 0 & 0 & 1             & 1                    \\ 
        0 & 0 & 1 & 1             & 1                    \\ 
        0 & 1 & 0 & 0             & 0                    \\ 
        0 & 1 & 1 & 1             & 1                    \\ 
        1 & 0 & 0 & 0             & 0                    \\ 
        1 & 0 & 1 & 1             & 1                    \\ 
        1 & 1 & 0 & 0             & 0                    \\ 
        1 & 1 & 1 & 1             & 1                    \\
        \hline
    \end{tabular}
	

\part{b}

    \begin{tabular}{|c|c|c|c|c|}
        \hline
        $x$ & $y$ & $z$ & $(x \rightarrow y) \wedge (z \rightarrow x)$ & $z \rightarrow y$ \\ \hline
        0 & 0 & 0 & 1                    & 1      \\ 
        0 & 0 & 1 & 0                    & 0      \\ 
        0 & 1 & 0 & 1                    & 1      \\ 
        0 & 1 & 1 & 0                    & 1      \\ 
        1 & 0 & 0 & 0                    & 1      \\ 
        1 & 0 & 1 & 0                    & 0      \\ 
        1 & 1 & 0 & 1                    & 1      \\ 
        1 & 1 & 1 & 1                    & 1      \\
        \hline
    \end{tabular}

$x = T$, $y = T$, $z = T$ satisfies both. 

\part{c}
Consider $x=0$, $y=1$, $z=1$. In this case, the first statement is false while the second is true.


\question{2}{Normal Forms}
\part{a}
\part{1}
\begin{eqnarray*}
\neg (p \leftrightarrow q)\\
\neg ((p \wedge q) \vee (\neg p \wedge \neg q))\\
\neg (p \wedge q) \wedge \neg (\neg p \wedge \neg q)\\
(\neg p \vee \neg q) \wedge (\neg \neg p \vee \neg \neg q)\\
(\neg p \vee \neg q) \wedge (p \vee q)\\
\end{eqnarray*}

\part{2}
\begin{eqnarray*}
\neg(p \vee (\neg q \wedge r))\\
\neg p \wedge \neg (\neg q \wedge r)\\
\neg p \wedge (\neg \neg q \vee \neg r)\\
\neg p \wedge (q \vee \neg r)\\
\end{eqnarray*}

\part{3}
\begin{eqnarray*}
\neg (\neg(p \rightarrow q) \rightarrow (p \rightarrow \neg q))\\
\neg (\neg \neg(p \rightarrow q) \vee (p \rightarrow \neg q))\\
\neg ((p \rightarrow q) \vee (p \rightarrow \neg q))\\
\neg (p \rightarrow q) \wedge \neg (p \rightarrow \neg q)\\
\neg (\neg p \vee q) \wedge \neg (\neg p \vee \neg q)\\
(\neg \neg p \wedge \neg q) \wedge (\neg \neg p \wedge \neg \neg q)\\
(p \wedge \neg q) \wedge (p \wedge q)\\
\end{eqnarray*}

\part{b}
\part{1}
\begin{eqnarray*}
(p_1 \wedge q_1) \vee (p_2 \wedge \neg q_2) \vee (\neg p_3 \wedge \neg q_3)\\
\\
\equiv ((p_1 \vee p_2) \wedge (p_1 \vee \neg q_2) \wedge (q_1 \vee p_2) \wedge (q_1 \vee \neg q_2)) \vee (\neg p_3 \wedge \neg q_3)\\
\\
\equiv (((p_1 \vee p_2) \vee \neg p_3) \wedge ((p_1 \vee p_2) \vee \neg q_3) \wedge ((p_1 \vee \neg q_2) \vee \neg p_3)\\
\wedge ((q_1 \vee p_2) \vee \neg q_3) \wedge ((q_1 \vee \neg q_2) \vee \neg p_3) \wedge ((q_1 \vee \neg q_2) \vee \neg q_3)\\
\\
\equiv (p_1 \vee p_2 \vee \neg p_3) \wedge (p_1 \vee p_2 \vee \neg q_3) \wedge (p_1 \vee \neg q_2 \vee \neg p_3)\\
\wedge (q_1 \vee p_2 \vee \neg q_3) \wedge (q_1 \vee \neg q_2 \vee \neg p_3) \wedge (q_1 \vee \neg q_2 \vee \neg q_3)\\
\end{eqnarray*}

\part{2}
\begin{eqnarray*}
((p_1 \vee \neg p_2) \wedge (\neg p_3 \vee p_4)) \leftrightarrow (p_5 \wedge \neg p_6)\\
(\neg ((p_1 \vee \neg p_2) \wedge (\neg p_3 \vee p_4)) \vee (p_5 \wedge \neg p_6)) \wedge (((p_1 \vee \neg p_2) \wedge (\neg p_3 \vee p_4)) \vee \neg (p_5 \wedge \neg p_6))\\
((\neg (p_1 \vee \neg p_2) \vee \neg (\neg p_3 \vee p_4)) \vee (p_5 \wedge \neg p_6)) \wedge (((p_1 \vee \neg p_2) \wedge (\neg p_3 \vee p_4)) \vee (\neg p_5 \vee \neg \neg p_6))\\
(((\neg p_1 \wedge\neg \neg p_2) \vee (\neg \neg p_3 \wedge \neg p_4)) \vee (p_5 \wedge \neg p_6)) \wedge (((p_1 \vee \neg p_2) \wedge (\neg p_3 \vee p_4)) \vee (\neg p_5 \vee \neg \neg p_6))\\
(((\neg p_1 \wedge p_2) \vee (p_3 \wedge \neg p_4)) \vee (p_5 \wedge \neg p_6)) \wedge (((p_1 \vee \neg p_2) \wedge (\neg p_3 \vee p_4)) \vee (\neg p_5 \vee p_6))\\
((\neg p_1 \wedge p_2) \vee (p_3 \wedge \neg p_4) \vee (p_5 \wedge \neg p_6)) \wedge (((p_1 \vee \neg p_2) \wedge (\neg p_3 \vee p_4)) \vee \neg p_5 \vee p_6)\\
\end{eqnarray*}

\begin{eqnarray*}
g_1 \leftrightarrow (\neg p_1 \wedge p_2)\\
\equiv (\neg g_1 \vee \neg p_1) \wedge (\neg g_1 \vee p_2) \wedge (\neg \neg p_1 \vee \neg p_2 \vee g_1)\\
\equiv (\neg g_1 \vee \neg p_1) \wedge (\neg g_1 \vee p_2) \wedge (p_1 \vee \neg p_2 \vee g_1)\\
\end{eqnarray*}

\begin{eqnarray*}
g_2 \leftrightarrow (p_3 \wedge \neg p_4)\\
\equiv (\neg g_2 \vee p_3) \wedge (\neg g_2 \vee \neg p_4) \wedge (\neg p_3 \vee \neg \neg p_4 \vee g_2)\\
\equiv (\neg g_2 \vee p_3) \wedge (\neg g_2 \vee \neg p_4) \wedge (\neg p_3 \vee p_4 \vee g_2)\\
\end{eqnarray*}

\begin{eqnarray*}
g_3 \leftrightarrow (p_5 \wedge \neg p_6)\\
\equiv (\neg g_3 \vee p_5) \wedge (\neg g_3 \vee \neg p_6) \wedge (\neg p_5 \vee \neg \neg p_6 \vee g_3)\\
\equiv (\neg g_3 \vee p_5) \wedge (\neg g_3 \vee \neg p_6) \wedge (\neg p_5 \vee p_6 \vee g_3)\\
\end{eqnarray*}

\begin{eqnarray*}
g_4 \leftrightarrow (p_1 \wedge \neg p_2)\\
\equiv (\neg g_4 \vee p_1) \wedge (\neg g_4 \vee \neg p_2) \wedge (\neg p_1 \vee \neg \neg p_2 \vee g_4)\\
\equiv (\neg g_4 \vee p_1) \wedge (\neg g_4 \vee \neg p_2) \wedge (\neg p_1 \vee p_2 \vee g_4)\\
\end{eqnarray*}

\begin{eqnarray*}
g_5 \leftrightarrow (\neg p_3 \wedge p_4)\\
\equiv (\neg g_5 \vee \neg p_3) \wedge (\neg g_5 \vee p_4) \wedge (\neg \neg p_3 \vee \neg p_4 \vee g_5)\\
\equiv (\neg g_5 \vee \neg p_3) \wedge (\neg g_5 \vee p_4) \wedge (p_3 \vee \neg p_4 \vee g_5)\\
\end{eqnarray*}

\begin{eqnarray*}
((p_1 \vee \neg p_2) \wedge (\neg p_3 \vee p_4)) \leftrightarrow (p_5 \wedge \neg p_6)\\
\equiv ((\neg p_1 \wedge p_2) \vee (p_3 \wedge \neg p_4) \vee (p_5 \wedge \neg p_6)) \wedge (((p_1 \vee \neg p_2) \wedge (\neg p_3 \vee p_4)) \vee \neg p_5 \vee p_6)\\
\equiv (g_1 \vee g_2 \vee g_3) \wedge (g_4 \vee \neg p_5 \vee p_6) \wedge (g_5 \vee \neg p_5 \vee p_6))\\
\wedge (\neg g_1 \vee \neg p_1) \wedge (\neg g_1 \vee p_2) \wedge (p_1 \vee \neg p_2 \vee g_1) \\
\wedge (\neg g_2 \vee p_3) \wedge (\neg g_2 \vee \neg p_4) \wedge (\neg p_3 \vee p_4 \vee g_2) \\
\wedge (\neg g_3 \vee p_5) \wedge (\neg g_3 \vee \neg p_6) \wedge (\neg p_5 \vee p_6 \vee g_3) \\
\wedge (\neg g_4 \vee p_1) \wedge (\neg g_4 \vee \neg p_2) \wedge (\neg p_1 \vee p_2 \vee g_4) \\
\wedge (\neg g_5 \vee \neg p_3) \wedge (\neg g_5 \vee p_4) \wedge (p_3 \vee \neg p_4 \vee g_5)\\
\end{eqnarray*}

\question{3}{Satisfiability Checking}
\part{a}
Consider clause $q$.
We can say $C_1 \equiv p \vee q$ by the first clause, and so, $C'_1 \equiv p$. Also, $C_2 \equiv F \vee \neg q$ by the second clause, and so, $C'_2 \equiv F$. This means we can replace clauses 1 and 2 with the clause $C'_1 \vee C'_2 \equiv p \vee F \equiv p$ by the resolution rule.

This makes our set of clauses: $\{p, \neg p \vee q \vee \neg l, r \vee l, \neg r \vee \neg l, \neg r \vee \neg p\}$.

Next, perform the same steps as above. $C_1 \equiv F \vee P$, so $C'_1 = F$, and $C_2 \equiv q \vee \neg l \vee \neg p$, so $C'_2 \equiv q \vee \neg l$. We can replace those two clauses by the resolution rule with $C'_1 \vee C'_2 \equiv F \vee q \vee \neg l \equiv q \vee \neg l$.

This makes our set of clauses: $\{q \vee \neg l,  r \vee l, \neg r \vee \neg l, \neg r \vee \neg p\}$.

Perform the steps again. $C_1 \equiv q \vee \neg l$, so $C'_1 = q$ and $C_2 \equiv r \vee l$, so $C'_2 \equiv r$. By the resolution rule, $C'_1 \vee C'_2 \equiv q \vee r$.

This makes our set of clauses: $\{q \vee r, \neg r \vee \neg l, \neg r \vee \neg p\}$.

Perform the steps again. $C_1 \equiv q \vee r$, so $C'_1 = q$ and $C_2 \equiv \neg r \vee \neg l$, so $C'_2 \equiv \neg l$. By the resolution rule, $C'_1 \vee C'_2 \equiv q \vee \neg l$.

This makes our set of clauses: $\{q \vee \neg l, \neg r \vee \neg p\}$.

Reconsider our original set of clauses: $\{p \vee q,, \neg q \neg p \vee q \vee \neg l, r \vee l, \neg r \vee \neg l, \neg r \vee \neg p\}$.

Notice that the second clause consists of only $\neg q$. This means that to satisfy it, $q = F$. Looking at our reduced clause set, the first clause states $q \vee \neg l$, and since $q = F$, this means $\neg l = T \Rightarrow l = F$. Back to our original, we know that $p = T$ (by clause 1 and $q = F$), and so $r = F$ (by clause 6 and $p = T$), and so $l = T$ (by clause 4 and $r = F$). However, recall that we had deduced $l = F$ in our reduced form. We have reached a contradiction, and thus, the original set of clauses is unsatisfiable. 

\part{b}
\part{1}
\begin{eqnarray*}
||\bar a \vee b, \bar c \vee d, \bar e \vee \bar f, f \vee \bar e \vee \bar b \Rightarrow (Decide)\\
a||\bar a \vee b, \bar c \vee d, \bar e \vee \bar f, f \vee \bar e \vee \bar b \Rightarrow (UnitProp)\\
a,b_{\bar a \vee b}||\bar a \vee b, \bar c \vee d, \bar e \vee \bar f, f \vee \bar e \vee \bar b \Rightarrow (Decide)\\
a, b_{\bar a \vee b}, c||\bar a \vee b, \bar c \vee d, \bar e \vee \bar f, f \vee \bar e \vee \bar b \Rightarrow (UnitProp)\\
a, b_{\bar a \vee b}, c, d_{\bar c \vee d}||\bar a \vee b, \bar c \vee d, \bar e \vee \bar f, f \vee \bar e \vee \bar b \Rightarrow (Decide)\\
a, b_{\bar a \vee b}, c, d_{\bar c \vee d}, e||\bar a \vee b, \bar c \vee d, \bar e \vee \bar f, f \vee \bar e \vee \bar b \Rightarrow (UnitProp)\\
a, b_{\bar a \vee b}, c, d_{\bar c \vee d}, e||\bar a \vee b, \bar c \vee d, \bar e \vee \bar f, f \vee \bar e \vee \bar b \Rightarrow (Backjump)\\
a, b_{\bar a \vee b}, c, d_{\bar c \vee d}, e, \bar f_{\bar e \vee \bar f}||\bar a \vee b, \bar c \vee d, \bar e \vee \bar f, f \vee \bar e \vee \bar b \Rightarrow (Backjump)\\
a, c, d_{\bar c \vee d}, e, \bar f_{\bar e \vee \bar f}||\bar a \vee b, \bar c \vee d, \bar e \vee \bar f, f \vee \bar e \vee \bar b \Rightarrow (UnitProp)\\
a, \bar b_{f \vee \bar e \vee \bar b}, c, d_{\bar c \vee d}, e, \bar f_{\bar e \vee \bar f}||\bar a \vee b, \bar c \vee d, \bar e \vee \bar f, f \vee \bar e \vee \bar b \Rightarrow (Backjump)\\
\bar b_{f \vee \bar e \vee \bar b}, c, d_{\bar c \vee d}, e, \bar f_{\bar e \vee \bar f}||\bar a \vee b, \bar c \vee d, \bar e \vee \bar f, f \vee \bar e \vee \bar b \Rightarrow (UnitProp)\\
\bar a, \bar b_{f \vee \bar e \vee \bar b}, c, d_{\bar c \vee d}, e, \bar f_{\bar e \vee \bar f}||\bar a \vee b, \bar c \vee d, \bar e \vee \bar f, f \vee \bar e \vee \bar b \Rightarrow model \ found\\
\end{eqnarray*}

\part{2}
\begin{eqnarray*}
||a \vee \neg b, \neg a \vee \neg b, b \vee c, \neg c \vee b, a \vee d \Rightarrow (Decide)\\
a||a \vee \neg b, \neg a \vee \neg b, b \vee c, \neg c \vee b, a \vee d \Rightarrow (UnitProp)\\
a, \neg b_{\neg a \vee \neg b}||a \vee \neg b, \neg a \vee \neg b, b \vee c, \neg c \vee b, a \vee d \Rightarrow (UnitProp)\\
a, \neg b_{\neg a \vee \neg b}, c_{b \vee c}||a \vee \neg b, \neg a \vee \neg b, b \vee c, \neg c \vee b, a \vee d \Rightarrow (Conflict, Unsat)\\
\end{eqnarray*}

This is unsatisfiable because since the clauses $a \vee \bar b$ and $\bar a \vee \bar b$ reduces to $\bar b$, and $b \vee c$ and $b \vee \bar c$ reduces to just $b$. This means we have $b$ and $\bar b$, which is a contradiction. 

\end{document}
