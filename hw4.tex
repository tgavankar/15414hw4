% You should title the file with a .tex extension (hw1.tex, for example)
\documentclass[11pt]{article}

\usepackage{amsmath}
\usepackage{amssymb}
\usepackage{fancyhdr}
\usepackage{graphicx}

\oddsidemargin0cm
\topmargin-2cm         %I recommend adding these three lines to increase the
\textwidth16.5cm   %amount of usable space on the page (and save trees)
\textheight23.5cm  

\newcommand{\question}[2] {\vspace{.25in} \hrule\vspace{0.5em}
\noindent{\bf #1: #2} \vspace{0.5em}
\hrule \vspace{.10in}}
\renewcommand{\part}[1] {\vspace{.10in} {\bf (#1)}}

\newcommand{\myname}{Tanay Gavankar}
\newcommand{\myandrew}{tgavanka@andrew.cmu.edu}
\newcommand{\myhwnum}{Homework 4}

\setlength{\parindent}{0pt}
\setlength{\parskip}{5pt plus 1pt}

\pagestyle{fancyplain}
\lhead{\fancyplain{}{\textbf{HW\myhwnum}}}          % Note the different brackets!
\rhead{\fancyplain{}{\myname\\ \myandrew}}
\chead{\fancyplain{}{15-414}}

\begin{document}

\medskip                            % Skip a "medium" amount of space
                                   % (latex determines what medium is)
                                   % Also try: \bigskip, \littleskip




\thispagestyle{plain}
\begin{center}                      % Center the following lines
{\Large 11-414 Assignment \myhwnum} \\
\myname \\
\myandrew \\
Due: 11/14/12 \\
\end{center}




\question{1}{Classic Propositional Logic}
\part{a}

    \begin{tabular}{|c|c|c|c|c|}
        \hline
        x & y & z & (x \/ y) -> z & (x -> z) /\ (y -> z) \\ \hline
        0 & 0 & 0 & 1             & 1                    \\ 
        0 & 0 & 1 & 1             & 1                    \\ 
        0 & 1 & 0 & 0             & 0                    \\ 
        0 & 1 & 1 & 1             & 1                    \\ 
        1 & 0 & 0 & 0             & 0                    \\ 
        1 & 0 & 1 & 1             & 1                    \\ 
        1 & 1 & 0 & 0             & 0                    \\ 
        1 & 1 & 1 & 1             & 1                    \\
        \hline
    \end{tabular}


    \begin{tabular}{|c|c|c|c|c|}
        \hline
        x & y & z & (x -> y) /\ (z -> x) & z -> y \\ \hline
        0 & 0 & 0 & 1                    & 1      \\ 
        0 & 0 & 1 & 0                    & 0      \\ 
        0 & 1 & 0 & 1                    & 1      \\ 
        0 & 1 & 1 & 0                    & 1      \\ 
        1 & 0 & 0 & 0                    & 1      \\ 
        1 & 0 & 1 & 0                    & 0      \\ 
        1 & 1 & 0 & 1                    & 1      \\ 
        1 & 1 & 1 & 1                    & 1      \\
        \hline
    \end{tabular}

$x = T$, $y = T$, $z = T$ satisfies both. 

\part{b}

\part{c}


\question{2}{Normal Forms}
\part{a}
\part{1}

\part{2}

\part{3}

\part{b}
\part{1}
\part{2}

\question{3}{Satisfiability Checking}


\end{document}
