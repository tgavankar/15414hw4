% You should title the file with a .tex extension (hw1.tex, for example)
\documentclass[11pt]{article}

\usepackage{amsmath}
\usepackage{amssymb}
\usepackage{fancyhdr}
\usepackage{graphicx}

\oddsidemargin0cm
\topmargin-2cm         %I recommend adding these three lines to increase the
\textwidth16.5cm   %amount of usable space on the page (and save trees)
\textheight23.5cm  

\newcommand{\question}[2] {\vspace{.25in} \hrule\vspace{0.5em}
\noindent{\bf #1: #2} \vspace{0.5em}
\hrule \vspace{.10in}}
\renewcommand{\part}[1] {\vspace{.10in} {\bf (#1)}}

\newcommand{\myname}{Tanay Gavankar}
\newcommand{\myandrew}{tgavanka@andrew.cmu.edu}
\newcommand{\myhwnum}{Homework 4}

\setlength{\parindent}{0pt}
\setlength{\parskip}{5pt plus 1pt}

\pagestyle{fancyplain}
\lhead{\fancyplain{}{\textbf{HW\myhwnum}}}          % Note the different brackets!
\rhead{\fancyplain{}{\myname\\ \myandrew}}
\chead{\fancyplain{}{15-414}}

\begin{document}

\medskip                            % Skip a "medium" amount of space
                                   % (latex determines what medium is)
                                   % Also try: \bigskip, \littleskip




\thispagestyle{plain}
\begin{center}                      % Center the following lines
{\Large 11-414 Assignment \myhwnum} \\
\myname \\
\myandrew \\
Due: 11/14/12 \\
\end{center}




\question{1}{Classic Propositional Logic}
\part{a}

    \begin{tabular}{|c|c|c|c|c|}
        \hline
        $x$ & $y$ & $z$ & $(x \vee y) \rightarrow z$ & $(x \rightarrow z) \wedge (y \rightarrow z)$ \\ \hline
        0 & 0 & 0 & 1             & 1                    \\ 
        0 & 0 & 1 & 1             & 1                    \\ 
        0 & 1 & 0 & 0             & 0                    \\ 
        0 & 1 & 1 & 1             & 1                    \\ 
        1 & 0 & 0 & 0             & 0                    \\ 
        1 & 0 & 1 & 1             & 1                    \\ 
        1 & 1 & 0 & 0             & 0                    \\ 
        1 & 1 & 1 & 1             & 1                    \\
        \hline
    \end{tabular}
	

\part{b}

    \begin{tabular}{|c|c|c|c|c|}
        \hline
        $x$ & $y$ & $z$ & $(x \rightarrow y) \wedge (z \rightarrow x)$ & $z \rightarrow y$ \\ \hline
        0 & 0 & 0 & 1                    & 1      \\ 
        0 & 0 & 1 & 0                    & 0      \\ 
        0 & 1 & 0 & 1                    & 1      \\ 
        0 & 1 & 1 & 0                    & 1      \\ 
        1 & 0 & 0 & 0                    & 1      \\ 
        1 & 0 & 1 & 0                    & 0      \\ 
        1 & 1 & 0 & 1                    & 1      \\ 
        1 & 1 & 1 & 1                    & 1      \\
        \hline
    \end{tabular}

$x = T$, $y = T$, $z = T$ satisfies both. 

\part{c}
Consider $x=F$, $y=1$, $z=1$. In this case, the first statement is false while the second is true.


\question{2}{Normal Forms}
\part{a}
\part{1}
\begin{eqnarray*}
\neg (p \leftrightarrow q)\\
\neg ((p \wedge q) \vee (\neg p \wedge \neg q))\\
\neg (p \wedge q) \wedge \neg (\neg p \wedge \neg q)\\
(\neg p \vee \neg q) \wedge (\neg \neg p \vee \neg \neg q)\\
(\neg p \vee \neg q) \wedge (p \vee q)\\
\end{eqnarray*}

\part{2}
\begin{eqnarray*}
\neg(p \vee (\neg q \wedge r))\\
\neg p \wedge \neg (\neg q \wedge r)\\
\neg p \wedge (\neg \neg q \vee \neg r)\\
\neg p \wedge (q \vee \neg r)\\
\end{eqnarray*}

\part{3}
\begin{eqnarray*}
\neg (\neg(p \rightarrow q) \rightarrow (p \rightarrow \neg q))\\
\neg (\neg \neg(p \rightarrow q) \vee (p \rightarrow \neg q))\\
\neg ((p \rightarrow q) \vee (p \rightarrow \neg q))\\
\neg (p \rightarrow q) \wedge \neg (p \rightarrow \neg q)\\
\neg (\neg p \vee q) \wedge \neg (\neg p \vee \neg q)\\
(\neg \neg p \wedge \neg q) \wedge (\neg \neg p \wedge \neg \neg q)\\
(p \wedge \neg q) \wedge (p \wedge q)\\
\end{eqnarray*}

\part{b}
\part{1}
\begin{eqnarray*}
(p_1 \wedge q_1) \vee (p_2 \wedge \neg q_2) \vee (\neg p_3 \wedge \neg q_3)\\
\\
\equiv ((p_1 \vee p_2) \wedge (p_1 \vee \neg q_2) \wedge (q_1 \vee p_2) \wedge (q_1 \vee \neg q_2)) \vee (\neg p_3 \wedge \neg q_3)\\
\\
\equiv (((p_1 \vee p_2) \vee \neg p_3) \wedge ((p_1 \vee p_2) \vee \neg q_3) \wedge ((p_1 \vee \neg q_2) \vee \neg p_3)\\
\wedge ((q_1 \vee p_2) \vee \neg q_3) \wedge ((q_1 \vee \neg q_2) \vee \neg p_3) \wedge ((q_1 \vee \neg q_2) \vee \neg q_3)\\
\\
\equiv ((p_1 \vee p_2 \vee \neg p_3) \wedge (p_1 \vee p_2 \vee \neg q_3) \wedge (p_1 \vee \neg q_2 \vee \neg p_3)\\
\wedge (q_1 \vee p_2 \vee \neg q_3) \wedge (q_1 \vee \neg q_2 \vee \neg p_3) \wedge (q_1 \vee \neg q_2 \vee \neg q_3)\\
\end{eqnarray*}

\part{2}
\begin{eqnarray*}
((p_1 \vee \neg p_2) \wedge (\neg p_3 \vee p_4)) \leftrightarrow (p_5 \wedge \neg p_6)\\
\end{eqnarray*}

\question{3}{Satisfiability Checking}


\end{document}
